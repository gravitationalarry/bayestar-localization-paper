\documentclass{iopart}
\usepackage{iopams}
\usepackage{aas_macros}
\usepackage[colorlinks]{hyperref}
\input{acronyms}

\begin{document}

\title[Rapid-response Bayesian sky localization]{Rapid-response Bayesian sky localization for electromagnetic follow-up of gravitational-wave candidates}
\author{Leo Singer and Larry Price}
\address{LIGO Laboratory, California Institute of Technology, Pasadena, CA 91125, USA}
\ead{\mailto{leo.singer@ligo.org}, \mailto{larry.price@ligo.org}}

\begin{abstract}
Description of rapid Bayesian sky localization algorithm
\end{abstract}

\section{Measurement uncertainty}
\cite{fairhurst:2009} estimated the uncertainty in the matched-filter measurement of the \ac{TOA} of a \ac{GW} signal by computing the scalar Fisher information.  We extend this result by computing the Fisher information matrix describing the uncertainties in the joint estimation of the signal amplitude and \ac{TOA}.

When a \ac{CBC} occurs, a detector's strain observation may be described by a linear combination of two orthogonal \ac{GW} quadratures, $a(t)$ and $b(t)$, and additive, zero\nobreakdashes-mean, \acl{WSS}, Gaussian measurement noise $n(t)$, with power spectral density $S(f)$.  Introducing the unknown, real, quadrature amplitudes, $A$ and $B$, and the unknown \ac{TOA}, $t_0$, we can write
%
\begin{equation}\label{eq:xoft}
	x(t) = A a(t - t_0) + B b(t - t_0) + n(t).
\end{equation}
%
Transformation of \Eref{eq:xoft} to the frequency domain converts the time delay by $t_0$ to a complex phase,
%
\begin{equation}\label{eq:xoff}
	\tilde{x}(f) = \left(A \tilde{a}(f) + B \tilde{b}(f)\right)e^{-2 \pi i f t_0} + \tilde{n}(f).
\end{equation}
%
The Fisher information matrix for a measurement $\mathbf{x}$ described by the unknown parameters $\boldsymbol{\theta}$ may be written as
%
\begin{equation}
	\mathcal{I}_{jk} = \mathrm{E} \, \left[
		-\frac{\partial^2 \log
			\mathcal{L}(\mathbf{x}; \boldsymbol{\theta})}
			{\partial \theta_j \theta_k}
	\right].
\end{equation}
%
First, we write the likelihood of the data $\tilde{x}(f)$ given the parameters $(A, B, t_0$):
%
\begin{equation}\label{eq:trigger-likelihood}
	\fl\mathcal{L}(x; A, B, t_0) = p(x | A, B, t_0)
		\propto \exp \left[
		- \frac{1}{2} \int_{-\infty}^\infty \frac{\left|\tilde{x}(f)
			- \left(A \tilde{a}(f)
			- B \tilde{b}(f)\right) e^{-2 \pi i f t_0}\right|^2}{S(f)} \, \mathrm{d}f
	\right].
\end{equation}
%
The normalization has been dropped because the Fisher matrix elements depend only on derivatives of the log\nobreakdashes-likelihood function and any factor that is independent of the parameters makes no contribution.%
%
\footnote{It would be cleaner to work in discrete time so that the Gaussian likelihood function can be correctly normalized.  Wide-sense stationary, continuous time processes don't really exist.}
%
From hereon, unless otherwise specified the limits of integration are always $(-\infty, \infty)$ and will be dropped.  Expanding the integrand, we can write the log\nobreakdashes-likelihood function up to an additive constant as
%
\begin{eqnarray}\label{eq:trigger-likelihood}
	\log \mathcal{L}(x; A, B, t_0) \sim
	&- \frac{1}{2} \int \frac{|\tilde{x}(f)|^2}{S(f)} \mathrm{d}f \\
	&+ A \, \mathrm{Re} \int \frac{\tilde{a}^*(f) \tilde{x}(f) e^{2 \pi i f t_0}}{S(f)} \mathrm{d}f \nonumber\\
	&+ B \, \mathrm{Re} \int \frac{\tilde{b}^*(f) \tilde{x}(f) e^{2 \pi i f t_0}}{S(f)} \mathrm{d}f\nonumber\\
	&+ A B \, \mathrm{Re} \int \frac{\tilde{a}^*(f) \tilde{b}(f)}{S(f)} \mathrm{d}f \nonumber\\
	&- \frac{1}{2} A^2 \int \frac{|\tilde{a}(f)|^2}{S(f)} \mathrm{d}f
	- \frac{1}{2} B^2 \int \frac{|\tilde{b}(f)|^2}{S(f)} \mathrm{d}f.
\end{eqnarray}
%
We assume that $\tilde{a}(f)$ and $\tilde{b}(f)$ are orthogonal and have the same norm,
%
\begin{equation}
	\int \frac{\tilde{a}^*(f) \tilde{b}(f)}{S(f)} \mathrm{d}f = 0,
	\int \frac{|\tilde{a}(f)|^2}{S(f)} \mathrm{d}f =
	\int \frac{|\tilde{b}(f)|^2}{S(f)} \mathrm{d}f =
	\sigma^2.
\end{equation}
%
After we apply these substitutions, and drop the $|\tilde{x}(f)|^2$ term that is independent of $A$, $B$, or $t_0$, we can write the log\nobreakdashes-likelihood function as
%
\begin{eqnarray}\label{eq:trigger-likelihood-simplified}
	\log \mathcal{L}(x; A, B, t_0) \sim
	& A \, \mathrm{Re} \int \frac{\tilde{a}^*(f) \tilde{x}(f) e^{2 \pi i f t_0}}{S(f)} \mathrm{d}f 
	&+ B \, \mathrm{Re} \int \frac{\tilde{b}^*(f) \tilde{x}(f) e^{2 \pi i f t_0} }{S(f)} \mathrm{d}f\nonumber\\
	&- \frac{1}{2} A^2 \sigma^2 - \frac{1}{2} B^2 \sigma^2.
\end{eqnarray}
%
The necessary partial derivatives are
%
\begin{eqnarray*}
	\frac{\partial^2 \log \mathcal{L}}{\partial A \partial B} &=
	\frac{\partial^2 \log \mathcal{L}}{\partial A \partial t_0} =
	\frac{\partial^2 \log \mathcal{L}}{\partial B \partial t_0} = 0 \\
	\frac{\partial^2 \log \mathcal{L}}{\partial A^2} &=
	\frac{\partial^2 \log \mathcal{L}}{\partial B^2} = - \sigma^2 \\
	\frac{\partial^2 \log \mathcal{L}}{\partial {t_0}^2} &=
	2 \pi i A \int \frac{\left(\tilde{a}^*(f) \tilde{x}(f) e^{2 \pi i f t_0} - \tilde{a}(f) \tilde{x}^*(f) e^{-2 \pi i f t_0}\right)}{S(f)} f \mathrm{d}f \\
	&+ 2 \pi i B \int \frac{\left(\tilde{b}^*(f) \tilde{x}(f) e^{2 \pi i f t_0} - \tilde{b}(f) \tilde{x}^*(f) e^{-2 \pi i f t_0}\right)}{S(f)} f \mathrm{d}f \\
	&- (2 \pi t_0)^2 A \int \frac{\left(\tilde{a}^*(f) \tilde{x}(f) e^{2 \pi i f t_0} + \tilde{a}(f) \tilde{x}^*(f) e^{-2 \pi i f t_0}\right)}{S(f)} f^2 \mathrm{d}f \\
	&- (2 \pi t_0)^2 B \int \frac{\left(\tilde{b}^*(f) \tilde{x}(f) e^{2 \pi i f t_0} + \tilde{b}(f) \tilde{x}^*(f) e^{-2 \pi i f t_0}\right)}{S(f)} f^2 \mathrm{d}f.
\end{eqnarray*}
%
Finally, taking the expectation values of the partial derivatives, we find
%
\begin{eqnarray*}
	\mathcal{I}_{A,B} &= \mathcal{I}_{A,t_0} = \mathcal{I}_{B,t_0} = 0 \\
	\mathcal{I}_{A,A} &= \mathcal{I}_{B,B} = \sigma^2 \\
	\mathcal{I}_{t_0,t_0} &=
	- 2 \pi i A^2 \int \frac{|\tilde{a}(f)|^2}{S(f)} f\mathrm{d}f
	- 2 \pi i B^2 \int \frac{|\tilde{B}(f)|^2}{S(f)} f\mathrm{d}f \\
	&+ (2 \pi t_0)^2 \left(A^2 \int \frac{|\tilde{a}(f)|^2}{S(f)} f^2\mathrm{d}f +
	B^2 \int \frac{|\tilde{b}(f)|^2}{S(f)} f\mathrm{d}f \right) \\
	&+ (2\pi t_0)^2 2 A B \, \mathrm{Re} \int \frac{\tilde{a}^*(f)\tilde{b}(f)}{S(f)} f^2\mathrm{d}f.
\end{eqnarray*}
%
The first two terms of $\mathcal{I}_{t_0,t_0}$ vanish because the integrands $|\tilde{a}(f)|^2 f / S(f)$ and $|\tilde{b}(f)|^2 f / S(f)$ are odd about $f = 0$.  We assume that $\tilde{a}(f)$ and $\tilde{b}(f)$ are orthogonal with respect to $\emph{any}$ real weight, including both $S(f)$ and $S(f) / f^2$, so the last term in $\mathcal{I}_{t_0,t_0}$ vanishes as well.  Further, we make the substituions
%
\begin{eqnarray*}
	\rho_A &= A \sigma \\
	\rho_B &= B \sigma \\
	f_\mathrm{rms} &= \left[ \int \frac{|\tilde{a}(f)|^2}{S(f)} f^2\mathrm{d}f \right] ^ {1/2} \left[\int \frac{|\tilde{a}(f)|^2}{S(f)} \mathrm{d}f \right] ^ {-1/2} \\
	&=  \left[ \int \frac{|\tilde{b}(f)|^2}{S(f)} f^2\mathrm{d}f \right] ^ {1/2} \left[\int \frac{|\tilde{b}(f)|^2}{S(f)} \mathrm{d}f \right] ^ {-1/2}.
\end{eqnarray*}
%
Then, we have
%
\begin{eqnarray*}
	\mathcal{I}_{\rho_A,\rho_B} &= \mathcal{I}_{\rho_A,t_0} = \mathcal{I}_{\rho_B,t_0} = 0 \\
	\mathcal{I}_{\rho_A,\rho_A} &= \mathcal{I}_{\rho_B,\rho_B} = 1 \\
	\mathcal{I}_{t_0,t_0} &= (2 \pi t_0 f_\mathrm{rms})^2.
\end{eqnarray*}
%
The Cram\'{e}r-Rao inequality says that the inverse of the Fisher information matrix is a lower bound on the covariance of any unbiased estimator.  The fisher matrix for this estimation problem has been shown to be diagonal, so we can write
%
\begin{eqnarray}
	\mathrm{var}\, (\hat{\rho}_B) &\geq {\mathcal{I}_{\rho_A,\rho_A}}^{-1} = 1 \\
	\mathrm{var}\, (\hat{\rho}_B) &\geq {\mathcal{I}_{\rho_B,\rho_B}}^{-1} = 1 \\
	\mathrm{var}\, (\hat{t}_0) &\geq {\mathcal{I}_{t_0,t_0}}^{-1} = (2 \pi t_0 f_\mathrm{rms})^{-2}.
\end{eqnarray}


\ack Source code for \ac{BAYESTAR} is available on the \acs{LIGO} \acl{DASWG} web site at \url{http://www.lsc-group.phys.uwm.edu/daswg/projects/bayestar.html}.

Some of the results in this paper have been derived using HEALPix \cite{healpix}.

\acs{LIGO} was constructed by the California Institute of Technology and Massachusetts Institute of Technology with funding from the \ac{NSF} and operates under cooperative agreement PHY\nobreakdashes-0107417.  Some results were produced on the NEMO computing cluster operated by the Center for Gravitation and Cosmology at University of Wisconsin\nobreakdashes--Milwaukee under \ac{NSF} Grants PHY\nobreakdashes-0923409 and PHY\nobreakdashes-0600953.  This research is supported by the \ac{NSF} through a Graduate Research Fellowship to L.S.  This paper has \acs{LIGO} Document Number \acs{LIGO}\nobreakdashes-PXXXXXXX\nobreakdashes-vX.


\section*{References}
\bibliographystyle{iopart-num}
\bibliography{apj-jour,telescope}

\end{document}
